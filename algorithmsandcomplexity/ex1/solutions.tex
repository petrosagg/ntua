\section{Problem 1}
According to Sipser~\cite{sipser}...

\section{Problem 2}
We know that $1 + 1 = 2$, therefore:

\begin{align*}
  1 + 1 &= 2 \\
  2 + 2 &= 4 \\
  4 + 4 &= 8
\end{align*}

\section{Problem 3}

Our solution is depicted in Algorithm~\ref{alg:problem-3}.

\import{./}{algorithms/problem3}

\section{Problem 4}

\begin{lemma}
  All circles are round.
\end{lemma}
\begin{proof}
  Suppose, towards a contradiction, that there exists a circle $C$
  which is not round. Because it is not round, it must have some
  corner $Z$ on it. However, this contradicts the fact that it is
  a circle.
  \Qed
\end{proof}

\begin{theorem}\label{thm:blue}
  For all $n \in \mathbb{N}$, the number $n$ is blue.
\end{theorem}
\begin{proof}
  We know that $n = 0$ is blue.
  Suppose $n \in \mathbb{N}$ is blue. We will prove
  that $n + 1$ is also blue. We know that because the two numbers are touching,
  they must be the same color. Therefore $n + 1$ is also blue.
  By induction, the statement follows.
  \Qed
\end{proof}

\begin{corollary}
  All natural numbers have the same color.
\end{corollary}
\begin{proof}
  The corollary follows directly from Theorem~\ref{thm:blue}.
  \Qed
\end{proof}

\section{Problem 5}

The solution is illustrated in Figure~\ref{fig:problem-5}.

\begin{figure*}
  \centering
  \includegraphics[width=\textwidth,keepaspectratio]{figures/problem5.png}
  \caption{The solution to Problem 5.}
  \label{fig:problem-5}
\end{figure*}